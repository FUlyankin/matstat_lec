%!TEX TS-program = xelatex
\documentclass[12pt, a4paper, oneside]{article}

% Можно вставить разную преамбулу
\input{preamble}

\title{
\begin{center} 
\includegraphics[width=0.99\textwidth]{logo.png}
\end{center}

Посиделка 1: схема статистики}
\date{ } %\today}

% Если делаешь конспект, вписывай своё имя прямо сюда!
\author{Ульянкин Ппилиф \thanks{\url{https://github.com/FUlyankin/matstat\_lec}}}

\begin{document} % Конец преамбулы, начало файла

\maketitle

\epigraph{Пафосная наркоманская цитата}{\textit{Автор}}

В машинном обучении довольно часто приходится строить метрики качества разных моделей. Модели надо как-то сравнивать между собой. Делать это по точечным значениям метрик на тестовой выборке --- не самая лучшая идея. Результат сильно зависит от разбиения на трэйн и тест. 



\section{Зачем мы учим тервер}

\end{document}



Для accuracy: 

https://machinelearningmastery.com/confidence-intervals-for-machine-learning/

тут полезные ссылки

https://stats.stackexchange.com/questions/363382/confidence-interval-of-precision-recall-and-f1-score

http://users.stat.ufl.edu/~aa/articles/agresti_coull_1998.pdf

Текущий мой способ: https://stats.stackexchange.com/questions/132684/how-to-calculate-confidence-intervals-for-precision-recall-from-a-signal-dete

http://citeseerx.ist.psu.edu/viewdoc/download?doi=10.1.1.64.142&rep=rep1&type=pdf