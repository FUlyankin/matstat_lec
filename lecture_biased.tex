%!TEX TS-program = xelatex
\documentclass[12pt, a4paper, oneside]{article}

% Можно вставить разную преамбулу
\input{preamble}

\title{
\begin{center} 
\includegraphics[width=0.99\textwidth]{logo.png}
\end{center}

Посиделка 1: схема статистики}
\date{ } %\today}

% Если делаешь конспект, вписывай своё имя прямо сюда!
\author{Ульянкин Ппилиф \thanks{\url{https://github.com/FUlyankin/matstat\_lec}}}

\begin{document} % Конец преамбулы, начало файла

\maketitle

\epigraph{You may hate the dictator, but something... far worse is gonna fill that void if you depose of him. I've lived a million lifetimes. I've gone through every, every scenario. TVA is the only way.}{\textit{Nathaniel Richards a.k.a. He who remains}}


В этой лекции мы поговорим про то, зачем мы учим тервер. Мы обсудим, как с помощью него можно замоделировать своё невежество и попытаться найти ответы на вопросы, которые нас мучают. Обсудим проверку гипотез и построим несколько простых критериев, основанных на комбинаторике.


\section{Зачем мы учим тервер}


\end{document}