%!TEX TS-program = xelatex
\documentclass[12pt, a4paper, oneside]{article}

% Можно вставить разную преамбулу
\input{preamble}

\title{
\begin{center} 
\includegraphics[width=0.99\textwidth]{logo.png}
\end{center}

Посиделка 8: мощь средних}
\date{ } %\today}

% Если делаешь конспект, вписывай своё имя прямо сюда!
\author{Ульянкин Ппилиф \thanks{\url{https://github.com/FUlyankin/matstat_lec}}}

\begin{document} % Конец преамбулы, начало файла

\maketitle

\epigraph{\hfill Бесконечность --- не предел!}{\textit{Баз Лайтер (История игрушек, 1995)}}

В прошлой посиделке мы посмотрели на то, как ЦПТ и ЗБЧ позволяют нам оценивать неизвестные параметры, строить для них доверительные интервалы и проверять гипотезы. В этой посиделки мы продолжим эту линию и поговорим про то, как можно сконструировать АБ-тест для долей и средних с помощью асимптотического подхода.

\section{Что такое АБ-тестирование}

\todo[inline]{Красивое описание проблемы. О бизнесе, о том что каждый день тестируют убере и тп. }

% Есть Винни-Пух и он торгует мёдом. Винни-пух решил проверить что произойдет в его онлайн-магазине с покупками, если он сделает редизайн сайта. Редизайн ему подсказали UX-исследователи. Теперь ВП хочет проверить что будет с конверсиями пользователей в покупки. 

% - Treatment: $p_T$ - ей он показывает новый сайт $(5\% от всех пользователей)$
% - Conrol: $p_C$ - ей он показывает старый сайт 

% $H_0: p_T - p_C = 0$

% $H_A: p_T - p_C > 0$


\section{ }


% __1. Эксперимент__ 

% Предпосылки теста:

% - Мы должны отдавать себе отчёт, что все наблюдения должны быть независимы и одинаково распределены, чтобы тест на основе ЦПТ работал.
% - Много наблюдений, мы работаем с биг-датой.
% - У наших наблюдений конечная дисперсия (ни одно из наблюдений особо сильно не выделяется на фоне всех остальных). Иначе говоря у нас нес выбросов в данных. 



% __2. Союзник:__



% $$
% \bar X_n \overset{asy}{\sim} N \left( \mathbb{E}(X_i), \frac{Var(X_i)}{n} \right)
% $$

% __3. Данные и модель:__

% Две выборки независят друг от друга, так как пользователь случайно относится к одной из двух групп. 

% \begin{equation*}
%     \begin{aligned}
%          & X^c_1, \ldots, X^c_{n_c} \sim idd \quad Bern(p_c) \\
%          & X^T_1, \ldots, X^T_{n_T} \sim idd \quad Bern(p_T) 
%     \end{aligned}
% \end{equation*}

% __4. Критерий для проверки__

% С помощью ЦПТ я могу выписать следующую логику: 

% \begin{equation*}
%     \begin{aligned}
%         & \bar X_{n_c} = \hat p_c \overset{asy}{\sim} N \left( p_c, \frac{ p_c\cdot (1 - p_c)}{n_c} \right) \approx  N \left( p_c, \frac{ \hat p_c\cdot (1 - \hat p_c)}{n_c} \right)  \\
%         & \bar X_{n_T} = \hat p_T \overset{asy}{\sim} N \left( p_T, \frac{ p_T\cdot (1 - p_T)}{n_T} \right) \approx N \left( p_T, \frac{ \hat p_T\cdot (1 - \hat p_T)}{n_T} \right)
%     \end{aligned}
% \end{equation*}


% $$
% \hat p_c - \hat p_T \overset{asy}{\sim} N \left( p_c - p_T, \frac{ \hat p_c\cdot (1 - \hat p_c)}{n_c} + \frac{ \hat p_T \cdot (1 - \hat p_T)}{n_T}  \right)
% $$


% А дальше мы говорим: "А пусть у нас верна нулевая гипотеза, мы верим в статус-кво и нам нужны доказательства что он нарушен".

% $$
% \hat p_c - \hat p_T \underset{H_0}{\overset{asy}{\sim}} N \left(0, \frac{ \hat p_c\cdot (1 - \hat p_c)}{n_c} + \frac{ \hat p_T \cdot (1 - \hat p_T)}{n_T}  \right)
% $$

% $$
% Z = \frac{\hat p_c - \hat p_T}{ \sqrt{ \frac{\hat p_c\cdot (1 - \hat p_c)}{n_c} + \frac{ \hat p_T \cdot (1 - \hat p_T)}{n_T}}} \underset{H_0}{\overset{asy}{\sim}} N (0, 1)
% $$

% Ечли у нас наблюдаемое значение $z_{obs}$ оказывается в хвосте распределения, это означает что расстоение между долями очень большое. Видимо, разница между ними действительно есть. 

% Понятное дело, что по анлогии это делается для любых средних! То есть вы можете попробовать предположить, что выборка пришла из какого-то другого распределения, понять как выглядит дисперсия и подготовить асимптотический критерий. 


% > Остаётся много вопросиков про то, а как правильно спланировать эксперимент. Например, сколько надо наблюдений? 

% # Планирование эксперимента



% Нам надо как-то увязать между собой следующие показатели: 

% - MDE (Minimal detectable effect)
% - ошибка 1 рода
% - ошибка 2 рода 
% - количество наблюдений $n_c,n_T$
% - как должны между собой соотноситься $n_c$ и $n_T$



% __Способы выбрать $q$__

% Наш бюджет на эксперимент всегда ограничен. Мы всего можем собрать $n$ наблюдений, из которых в группу воздействия попадёт $q \cdot n$, то есть $n_T = q \cdot n$ и $n_c = (1 - q) \cdot n$. Как выбрать $q$? 




% __1)__  От балды: $q = 0.5$.


% __2)__  Хочется, чтобы дисперсия была поменьше, то есть: 


% $$
% Var(\hat p_T - \hat p_c) = \frac{p_T \cdot (1 - p_T)}{q \cdot n} + \frac{p_c \cdot (1 - p_c)}{(1 - q) \cdot n}  \to \min_q
% $$

% Если мы решим эту задачу, тогда у нас поулчится

% $$
% q = \frac{\sqrt{p_T \cdot (1 - p_T)} }{\sqrt{p_T \cdot (1 - p_T)} + \sqrt{p_c \cdot (1 - p_c)} }
% $$


% Скорее всего, все революционные идеи уже сделаны. Скорее всего, наши группы в целом похожи. $MDE$ маленькое. То есть $q \approx 0.5.$ 

% Либо если вы верите, что ваши изменения революционные и перевернут мир, можно запустить дорогой двухшаговый эксперимент, где за первую неделю вы получите точечную оценку для $q$, а на второй неделе с выверенной пропорцией уже проверите гипотезу. 

% __3)__ Обычно есть продукт сам по себе. Обычно для АБ-теста от этого продукта отсекают маленькую долю людей, например $1\%$. Нужно просто понять хватает ли нам для детекции эффекта $1\%$ или надо увеличить долю. 

% __4)__  Выбрать $q$ так, чтобы минимизировать $\beta$. Ниже мы подумали над этой идеей и она совпала с идеей номер 2. 



% Давайте проведёмс вычисления, которые свяжут эти три показателя для ситуации, когда $q = 0.5$.

% Критерий для двусторонней альтернативы: 

% $$
% \left| \frac{\hat d - 0}{se(\hat d)} \right| > z_{1-\frac{\alpha}{2}} \quad \Rightarrow \quad \text{отвергаем } H_0
% $$



\end{document}
