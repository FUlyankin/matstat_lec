%!TEX TS-program = xelatex
\documentclass[12pt, a4paper, oneside]{article}

% Можно вставить разную преамбулу
\input{preamble}

\title{
\begin{center} 
\includegraphics[width=0.99\textwidth]{logo.png}
\end{center}

Посиделка 1: схема статистики}
\date{ } %\today}

% Если делаешь конспект, вписывай своё имя прямо сюда!
\author{Ульянкин Ппилиф \thanks{\url{https://github.com/FUlyankin/matstat\_lec}}}

\begin{document} % Конец преамбулы, начало файла

\maketitle

\epigraph{You may hate the dictator, but something... far worse is gonna fill that void if you depose of him. I've lived a million lifetimes. I've gone through every, every scenario. TVA is the only way.}{\textit{Nathaniel Richards a.k.a. He who remains}}


В этой лекции мы поговорим про то, зачем мы учим тервер. Мы обсудим, как с помощью него можно замоделировать своё невежество и попытаться найти ответы на вопросы, которые нас мучают. Обсудим проверку гипотез и построим несколько простых критериев, основанных на комбинаторике.


\section{Зачем мы учим тервер}


\section{Состоятельность}

% \begin{problem}{ } 
% Пусть у нас есть выборка $X_1, \ldots, X_n$ из какого-то распределения с математическим ожиданием $\mu$ и дисперсией $\sigma^2 < \infty$. Нужно понять к чему сходятся $\bar x$ и $\hat \sigma^2 = \overline{x^2} - \bar{x}^2,$ посчитанные по этой выборке.
% \end{problem} 

% \begin{sol}
% Со средним всё просто. Вспоминаем ЗБЧ. Он говорит нам, что при конечном втором моменте

% $$
% \bar x = \frac{X_1 + X_2 + \ldots + X_n}{n} \overset{p}{\to} \E(X_i) = \mu.
% $$

% Получается, что среднее --- это состоятельная оценка для математического ожидания. 

% Для того, чтобы понять к чему сходится выборочная дисперсия, нужно воспользоваться более широким арсеналом. Во-первых, по теореме Манна-Вальда 

% $$
% \bar{x}^2 \overset{p}{\to} \E(X_i)^2.
% $$

% Если сделать замену $Y_i = X_i^2,$ то снова можно воспользоваться ЗБЧ

% $$
% \overline{x^2} = \frac{X_1^2 + X_2^2 + \ldots + X_n^2}{n} = \frac{Y_1 + Y_2 + \ldots + Y_n}{n} \overset{p}{\to} \E(Y_i) = \E(X_i^2).
% $$

% Тут стоит заметить, что мы можем использовать ЗБЧ для $Y_i$ только в той ситуации, когда $\E(Y_i^2) = \E(X_i^4) < \infty.$ Мы будем довольно часто сталкиваться с этим условием в предпосылках разных моделей. \indef{На простецком оно означает, что в данных нет выбросов.}

% Мы нашли к чему сходятся оба значения. Осталось воспользоваться арифметическими свойствами предела, а именно тем, что предел разности равен разности пределов. Получается, что 

% $$
% \hat \sigma^2 = \bar{x^2} - \bar{x}^2 \overset{p}{\to} \E(X_i^2) - \E(X_i)^2 = \Var(X_i) = \sigma^2,
% $$

% то есть выборочная дисперсия --- состоятельная оценка для дисперсии.
% \end{sol} 

% На практике для проверки состоятельности оценки иногда пользуются \indef{достаточным условием Чебышёва.} 

% \begin{theorem}{\textbf{Достаточное условие Чебышёва}}
% Если $\E(\hat{\theta}) \to \theta$ и $\Var(\hat{\theta}) \to 0$ при $n \to \infty$, тогда оценка $\hat{\theta}$ состоятельная. 
% \end{theorem}

% Нужно понимать, что эти условия являются достаточными. То есть если они выполнены, то оценка состоятельная. Если они не выполнены, мы не можем сказать про оценку ничего конкретного. В такой ситуации нужно по-честному искать предел. Давайте посмотрим на искусственный пример, где оценка состоятельна, но условие Чебышёва не работает. 

% \begin{problem}{ } 
% Пусть оценка неизвестного параметра $\hat \theta$ имеет следующее распределение 

% \begin{center}
% \begin{tabular}{c|c|c|c}
% $\hat{\theta}_n$ & $\theta - 100^n$ & $\theta$ & $\theta + 100^n$  \\
% \hline
% $\PP(hat{\theta}_n = k)$ & $0.1^n$ & $1 - 2 \cdot 0.1^n$ & $0.1^n$ \\
% \end{tabular}
% \end{center}

% Будет ли оценка состоятельной? Работает ли здесь условие Чебышёва? 
% \end{problem} 

% \begin{sol}
% По табличке видно, что при $n \to \infty$ наше распределение вырождается в $\theta$. Это означает, что у нас есть сходимость по вероятности. При этом $\Var(\hat{\theta}_n) \to \infty$ и условие Чебышёва не выполняется. 
% \end{sol} % \begin{problem}{ } 
% Пусть у нас есть выборка $X_1, \ldots, X_n$ из какого-то распределения с математическим ожиданием $\mu$ и дисперсией $\sigma^2 < \infty$. Нужно понять к чему сходятся $\bar x$ и $\hat \sigma^2 = \overline{x^2} - \bar{x}^2,$ посчитанные по этой выборке.
% \end{problem} 

% \begin{sol}
% Со средним всё просто. Вспоминаем ЗБЧ. Он говорит нам, что при конечном втором моменте

% $$
% \bar x = \frac{X_1 + X_2 + \ldots + X_n}{n} \overset{p}{\to} \E(X_i) = \mu.
% $$

% Получается, что среднее --- это состоятельная оценка для математического ожидания. 

% Для того, чтобы понять к чему сходится выборочная дисперсия, нужно воспользоваться более широким арсеналом. Во-первых, по теореме Манна-Вальда 

% $$
% \bar{x}^2 \overset{p}{\to} \E(X_i)^2.
% $$

% Если сделать замену $Y_i = X_i^2,$ то снова можно воспользоваться ЗБЧ

% $$
% \overline{x^2} = \frac{X_1^2 + X_2^2 + \ldots + X_n^2}{n} = \frac{Y_1 + Y_2 + \ldots + Y_n}{n} \overset{p}{\to} \E(Y_i) = \E(X_i^2).
% $$

% Тут стоит заметить, что мы можем использовать ЗБЧ для $Y_i$ только в той ситуации, когда $\E(Y_i^2) = \E(X_i^4) < \infty.$ Мы будем довольно часто сталкиваться с этим условием в предпосылках разных моделей. \indef{На простецком оно означает, что в данных нет выбросов.}

% Мы нашли к чему сходятся оба значения. Осталось воспользоваться арифметическими свойствами предела, а именно тем, что предел разности равен разности пределов. Получается, что 

% $$
% \hat \sigma^2 = \bar{x^2} - \bar{x}^2 \overset{p}{\to} \E(X_i^2) - \E(X_i)^2 = \Var(X_i) = \sigma^2,
% $$

% то есть выборочная дисперсия --- состоятельная оценка для дисперсии.
% \end{sol} 

% На практике для проверки состоятельности оценки иногда пользуются \indef{достаточным условием Чебышёва.} 

% \begin{theorem}{\textbf{Достаточное условие Чебышёва}}
% Если $\E(\hat{\theta}) \to \theta$ и $\Var(\hat{\theta}) \to 0$ при $n \to \infty$, тогда оценка $\hat{\theta}$ состоятельная. 
% \end{theorem}

% Нужно понимать, что эти условия являются достаточными. То есть если они выполнены, то оценка состоятельная. Если они не выполнены, мы не можем сказать про оценку ничего конкретного. В такой ситуации нужно по-честному искать предел. Давайте посмотрим на искусственный пример, где оценка состоятельна, но условие Чебышёва не работает. 

% \begin{problem}{ } 
% Пусть оценка неизвестного параметра $\hat \theta$ имеет следующее распределение 

% \begin{center}
% \begin{tabular}{c|c|c|c}
% $\hat{\theta}_n$ & $\theta - 100^n$ & $\theta$ & $\theta + 100^n$  \\
% \hline
% $\PP(hat{\theta}_n = k)$ & $0.1^n$ & $1 - 2 \cdot 0.1^n$ & $0.1^n$ \\
% \end{tabular}
% \end{center}

% Будет ли оценка состоятельной? Работает ли здесь условие Чебышёва? 
% \end{problem} 

% \begin{sol}
% По табличке видно, что при $n \to \infty$ наше распределение вырождается в $\theta$. Это означает, что у нас есть сходимость по вероятности. При этом $\Var(\hat{\theta}_n) \to \infty$ и условие Чебышёва не выполняется. 
% \end{sol} 

доказать чебышева как во 2 главе тут  http://personal.psu.edu/drh20/asymp/fall2006/lectures/

ближе к концу


\end{document}